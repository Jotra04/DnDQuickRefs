\documentclass[a4paper]{article}
\usepackage[a4paper, total={6in, 10in}]{geometry}
\author{Jonas Desloovere}
\title{Dual wielding}
\thispagestyle{empty}
\begin{document}
\part*{\begin{center}
Dual wielding\\
	\large DnD 5e Quickref
\end{center}
}
\section{Basic rules \small (PHB p.195)}
\begin{itemize}
\item You must hold a light melee weapon in each hand
\item You must take the Attack action with one of those to attack with the other
\item The second attack uses your bonus action
\item You don't add your ability modifier to the damage of the bonus attack, unless that modifier is negative
\item If either weapon has the thrown property, you can throw it instead of making a melee attack with it
\item Both attacks can target a different target and you can move between the attacks
\end{itemize}
\section{Broken down}
\begin{enumerate}
\item Hold a light melee weapon in each hand
\item Take the Attack action with one of the weapons
	\begin{itemize}
	\item Attack roll: 1d20 + prof + STR (finesse: DEX)
	\item Damage roll: damage die + STR (finesse: DEX)
	\end{itemize}
\item Take the Attack bonus action with the other weapon
	\begin{itemize}
	\item Attack roll: 1d20 + prof + STR (finesse: DEX)
	\item Damage roll: damage die (no modifiers)
	\end{itemize}
\end{enumerate}
\section{Some notes}
\begin{itemize}
\item When a feature (ex.\ extra attack) activates when you use your Action to attack, it does not activate on your bonus action attack
\item Features that activate 'on a hit' without explicitly requiring the Action (ex.\ sneak attack) can activate on your bonus action attack
\item The 'Two weapon fighting' style allows you to add your ability modifier to the damage roll of the bonus attack
\item You have the Bonus action required to make this bonus attack as long as you have a light melee weapon in each hand
\item Using this rule one can make 3 attacks per round, without any other features
	\begin{enumerate}
	\item On your turn using your Action
	\item On your turn using your Bonus action
	\item On another creature's turn as an opportunity attack using your Reaction
	\end{enumerate}
\end{itemize}
\end{document}
