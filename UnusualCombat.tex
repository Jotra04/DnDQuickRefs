\documentclass[a4paper]{article}
\usepackage[a4paper, total={6in, 10in}]{geometry}
\author{Jonas Desloovere}
\title{Unusual combat actions}
\thispagestyle{empty}
\begin{document}
\part*{\begin{center}
Unusual combat actions\\
	\large DnD 5e Quickref
\end{center}
}
\section{Item interactions \small (PHB p.190)}
You can interact with one item or feature of the environment for free, during your movement or action.
A second interaction requires an action.
\vspace{3mm}

Examples:
\begin{itemize}
\item Draw or sheathe your weapon
\item Open a door, pull a lever or extinguish a small flame
\item Take an item from a container or pick it up
\item Stuff something into your mouth or drink something
\item Hand someone something or accept something from someone
\end{itemize}
\section{Dropping prone \small (PHB p.190 + 292)}
\begin{itemize}
\item You can drop prone without any costs
\item Standing back up requires half your movement
\item You cannot stand up when your movement is 0
\item While prone, you can only crawl, costing 2 ft for every foot of movement and you have disadvantage on attack rolls
\item An attack against you has advantage when the attacker is within 5 ft otherwise the attack has disadvantage
\end{itemize}
\section{Search \small (PHB p.193)}
You stop paying attention to the combat and instead focus on finding something, depending on the nature of your search, your DM may have you roll a Perception (WIS) or Investigation (INT) check.
\section{Special melee attacks \small (PHB p.195)}
Grapple and shove:
\begin{itemize}
\item Replace one attack against a target within reach
\item The target must be no more than 1 size larger than you
\item Make an Athletics (STR) check against the opponents Athletics (STR) or Acrobatics (DEX) check instead of an attack roll
\end{itemize}
Grapple:
\begin{itemize}
\item You must have one hand free
\item On a success the target is grappled
\item While grappled the target can try to escape, forcing another grapple check
\item You can move and take the grappled target with you, doing so halves your speed, unless the creature is 2 or more sizes smaller than you
\item You can release the target whenever you want, without an action
\end{itemize}
Shove:
\begin{itemize}
\item On a success the target is knocked prone or pushed away 5 ft
\end{itemize}
\end{document}
